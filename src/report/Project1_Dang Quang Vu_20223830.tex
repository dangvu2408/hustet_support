\documentclass{article}
\usepackage[utf8]{vietnam}
\usepackage[OT1]{fontenc}
\usepackage[fontsize=13pt]{scrextend}
\usepackage[paperheight=29.7cm,paperwidth=21cm,right=2cm,left=3cm,top=2cm,bottom=2.5cm,twoside]{geometry}% Chuẩn A4, căn lề phải, trái, trên, dưới.
\usepackage{mathptmx}
\usepackage{amsmath}
\usepackage{graphicx} % Thư viện chèn ảnh
\usepackage{float} % Set vị trí chèn ảnh
\usepackage{tikz} % Thư viện tạo khung bìa
\usepackage{fontspec}
\setmonofont{Courier New}
\usetikzlibrary{calc} % Thư viện tikz
\usepackage{indentfirst} % Thư viện thụt đầu dòng
\usepackage{booktabs} % To thicken table lines
\usepackage{amssymb}

\renewcommand{\baselinestretch}{1.2} % Giãn dòng 1.2
\setlength{\parskip}{6pt} % Spacing after
\setlength{\parindent}{1cm} % Set khoảng cách thụt đầu dòng mỗi đoạn
\usepackage{titlesec} % Thư viện để set up các kiểu chữ
\usepackage{listings}
\usepackage{matlab-prettifier}
\setcounter{secnumdepth}{4} % 4 Heading
\titlespacing*{\section}{0pt}{0pt}{30pt} % Heading 1
\titleformat*{\section}{\fontsize{16pt}{0pt}\selectfont \bfseries \centering}

\titlespacing*{\subsection}{0pt}{10pt}{0pt} % Heading 2
\titleformat*{\subsection}{\fontsize{14pt}{0pt}\selectfont \bfseries}

\titlespacing*{\subsubsection}{0pt}{10pt}{0pt} % Heading 3
\titleformat*{\subsubsection}{\fontsize{13pt}{0pt}\selectfont \bfseries \itshape}

\titlespacing*{\paragraph}{0pt}{10pt}{0pt} % Heading 4
\titleformat*{\paragraph}{\fontsize{13pt}{0pt}\selectfont \itshape}

\renewcommand{\figurename}{\fontsize{12pt}{0pt}\selectfont \bfseries Figure}
\renewcommand{\thefigure}{\thesection.\arabic{figure}}
\usepackage[font=bf]{caption}
\captionsetup[figure]{labelsep=space}

\renewcommand{\tablename}{\fontsize{12pt}{0pt}\selectfont \bfseries Table}
\renewcommand{\thetable}{\thesection.\arabic{table}}
\captionsetup[table]{labelsep=space}

\usepackage{tabularx}
\newcolumntype{s}{>{\hsize=.3\hsize}X}
\newcolumntype{y}{>{\hsize=.4\hsize}X}
\newcolumntype{d}{>{\hsize=.1\hsize}X}
\newcolumntype{a}{>{\hsize=1.1\hsize}X}
\newcolumntype{g}{>{\hsize=5\hsize}X}
\renewcommand{\tabularxcolumn}[1]{>{\small}m{#1}}

\renewcommand{\theequation}{\thesection.\arabic{equation}} % Thay đổi đánh số phương trình mặc định
\newtheorem{theorem}{Định lý}[section]
\newtheorem{defn}[theorem]{Định nghĩa}
\newtheorem{corollary}[theorem]{Hệ quả}
\newtheorem{lemma}[theorem]{Bổ đề}
\usepackage{lipsum} % Thư viện tạo chữ linh tinh.

\usepackage[unicode]{hyperref}
\usepackage{colortbl}
\definecolor{LightCyan}{rgb}{0.88,1,1}
\usepackage{forloop}
\newcounter{loopcntr}
\newcommand{\rpt}[2][1]{\forloop{loopcntr}{0}{\value{loopcntr}<#1}{#2}}

\begin{document}
	\setmainfont{Times New Roman}
	\thispagestyle{empty}
	\begin{center}
		\vspace{-12pt}  \fontsize{14pt}{0pt}\selectfont ĐẠI HỌC BÁCH KHOA HÀ NỘI \\[6pt]
		\textbf{\fontsize{16pt}{0pt}\selectfont TRƯỜNG ĐIỆN - ĐIỆN TỬ}
		\vspace{1.75cm}
		\begin{figure}[H]
			\centering
			\includegraphics[height=4.25cm]{logoHUST.png}
		\end{figure}
		\vspace{1cm}
		\textbf{\fontsize{25pt}{0pt}\selectfont ĐỒ ÁN THIẾT KẾ I} 
		\vspace{0.5cm}
	\end{center}
	\begin{center}
		\textbf{\fontsize{22pt}{0pt}\selectfont Xây dựng website hỗ trợ học tập} \\
		\vspace{2.5cm}
		
		\textbf{\fontsize{18pt}{0pt}\selectfont ĐẶNG QUANG VŨ} \\[6pt]
		\fontsize{16pt}{0pt}\selectfont vu.dq223830@sis.hust.edu.vn \\[6pt]
		\vspace{0.75cm}
		\textbf{\fontsize{16pt}{0pt}\selectfont Ngành Điện tử - Viễn thông} \\[6pt]
		\textbf{\fontsize{16pt}{0pt}\selectfont Chuyên ngành Kỹ thuật Điện tử - Viễn thông} 
		\vspace{0.75cm}
		\begin{table}[H]
			\centering
			\begin{tabular}{l l l}
				\fontsize{16pt}{0pt}\selectfont \textbf{Giảng viên hướng dẫn:}    & \fontsize{16pt}{0pt}\selectfont ThS. Đinh Thị Nhung \vspace{6pt} & \_\_\_\_\_\_\_\_\_\_\_ \\ 
			\end{tabular}
		\end{table}
		\vspace{2.5cm}
		\fontsize{14pt}{0pt}\selectfont \textbf{Hà Nội, 6/2025}
	\end{center}
	\cleardoublepage
	\thispagestyle{empty}
	\begin{center}
		\vspace{-12pt}  \fontsize{14pt}{0pt}\selectfont ĐẠI HỌC BÁCH KHOA HÀ NỘI \\[6pt]
		\textbf{\fontsize{16pt}{0pt}\selectfont TRƯỜNG ĐIỆN - ĐIỆN TỬ}
		\vspace{1.75cm}
		\begin{figure}[H]
			\centering
			\includegraphics[height=4.25cm]{logoHUST.png}
		\end{figure}
		\vspace{1cm}
		\textbf{\fontsize{25pt}{0pt}\selectfont ĐỒ ÁN THIẾT KẾ I} 
		\vspace{0.5cm}
	\end{center}
	\begin{center}
		\textbf{\fontsize{22pt}{0pt}\selectfont Xây dựng website hỗ trợ học tập} \\
		\vspace{2.5cm}
		
		\textbf{\fontsize{18pt}{0pt}\selectfont ĐẶNG QUANG VŨ} \\[6pt]
		\fontsize{16pt}{0pt}\selectfont vu.dq223830@sis.hust.edu.vn \\[6pt]
		\vspace{0.75cm}
		\textbf{\fontsize{16pt}{0pt}\selectfont Ngành Điện tử - Viễn thông} \\[6pt]
		\textbf{\fontsize{16pt}{0pt}\selectfont Chuyên ngành Kỹ thuật Điện tử - Viễn thông} 
		\vspace{0.75cm}
		\begin{table}[H]
			\centering
			\begin{tabular}{l l l}
				\fontsize{16pt}{0pt}\selectfont \textbf{Giảng viên hướng dẫn:}    & \fontsize{16pt}{0pt}\selectfont ThS. Đinh Thị Nhung \vspace{6pt} &  \\ 
			\end{tabular}
		\end{table}
		\vspace{2.5cm}
		\fontsize{14pt}{0pt}\selectfont \textbf{Hà Nội, 6/2025}
	\end{center}
	\cleardoublepage
	\section*{LỜI NÓI ĐẦU}
	\thispagestyle{empty}
	Trong thời đại số hóa hiện nay, việc truy cập và sử dụng tài liệu học tập, nghiên cứu, cũng như các tài liệu chuyên ngành ngày càng trở nên thiết yếu đối với sinh viên. Tài liệu không chỉ là nguồn cung cấp kiến thức nền tảng, mà còn là công cụ hỗ trợ hiệu quả cho việc cập nhật kiến thức mới, rèn luyện kỹ năng tư duy phản biện, và thúc đẩy quá trình học tập.
	
	Tuy nhiên, cùng với sự bùng nổ về số lượng và chủng loại tài liệu, người dùng lại đối mặt với nhiều thách thức trong việc tìm kiếm, chọn lọc và truy cập tài liệu một cách hiệu quả. Các nền tảng tài liệu hiện nay thường phân tán trên nhiều hệ thống khác nhau, thiếu tính liên kết và đồng bộ. Trong khi một số hệ thống bị giới hạn phạm vi truy cập (chỉ dành cho một nhóm đối tượng cụ thể như sinh viên trong trường hoặc người có trả phí cao), thì các nền tảng mở lại thiếu sự tổ chức, kiểm duyệt và phân loại rõ ràng, gây nhiễu thông tin và ảnh hưởng đến trải nghiệm người dùng.
	
	Bên cạnh đó, không phải ai cũng có cùng một trình độ chuyên môn hoặc nhu cầu sử dụng tài liệu như nhau. Một hệ thống tài liệu lý tưởng cần phải phân tầng truy cập thông minh, cho phép người dùng ở các cấp độ (như người mới bắt đầu, người học nâng cao, hay chuyên gia) có thể tiếp cận với loại tài liệu phù hợp với trình độ và mục tiêu học tập của mình. Điều này không chỉ giúp tối ưu hóa trải nghiệm học tập, mà còn góp phần đảm bảo tính bảo mật, quản lý tốt hơn quyền truy cập và tránh lãng phí tài nguyên.
	
	Xuất phát từ những nhu cầu thực tiễn trên, việc xây dựng một hệ thống hỗ trợ quản lý và truy cập tài liệu học thuật và chuyên ngành theo hướng hiệu quả, có tổ chức và phân quyền rõ ràng là hết sức cần thiết. Đây sẽ là công cụ quan trọng nhằm kết nối người học với nguồn tài liệu giá trị, tạo điều kiện thúc đẩy việc học tập chủ động, phát triển năng lực cá nhân, đồng thời hỗ trợ tốt hơn cho các hoạt động giảng dạy và nghiên cứu trong môi trường học thuật hiện đại.
	\cleardoublepage
	
	\addtocontents{toc}{\protect\thispagestyle{empty}}
	\renewcommand{\contentsname}{MỤC LỤC}
	\tableofcontents 
	\thispagestyle{empty}
	\cleardoublepage
	
	\pagenumbering{roman} % Đánh số thứ tự la mã
	\section*{DANH MỤC KÝ HIỆU VÀ CHỮ VIẾT TẮT}
	\phantomsection \addcontentsline{toc}{section}{\numberline {} DANH MỤC KÝ HIỆU VÀ CHỮ VIẾT TẮT}
	
	\begin{tabular}{ l l }
		\hspace{1cm} AWGN & \hspace{4cm} Additive White Gaussian Noise \\  
		\hspace{1cm} BC & \hspace{4cm} Broadcast Channel    \\
		\hspace{1cm} BS  & \hspace{4cm} Base Station\\
		\hspace{1cm} CSI & \hspace{4cm} Channel State Information \\  
	\end{tabular}  
	
	\newpage
	
	\renewcommand{\listfigurename}{DANH MỤC HÌNH VẼ}
	{\let\oldnumberline\numberline
		\renewcommand{\numberline}{Hình~\oldnumberline}
		\listoffigures} 
	\phantomsection\addcontentsline{toc}{section}{\numberline {} DANH MỤC HÌNH VẼ}
	\newpage
	
	%Tạo danh mục bảng biểu.
	\renewcommand{\listtablename}{DANH MỤC BẢNG BIỂU}
	{\let\oldnumberline\numberline
		\renewcommand{\numberline}{Bảng~\oldnumberline}
		\listoftables}
	\phantomsection\addcontentsline{toc}{section}{\numberline {} DANH MỤC BẢNG BIỂU}
	\newpage
	
	\section*{TÓM TẮT ĐỒ ÁN}
	\phantomsection\addcontentsline{toc}{section}{\numberline {}TÓM TẮT ĐỒ ÁN}
	Trong bối cảnh số hóa mạnh mẽ hiện nay, nhu cầu truy cập và khai thác tài liệu học tập, nghiên cứu ngày càng gia tăng. Tuy nhiên, việc phân tán tài liệu trên nhiều nền tảng, thiếu tính tổ chức và không phân quyền truy cập hợp lý khiến người dùng gặp khó khăn trong việc tìm kiếm, sử dụng tài nguyên hiệu quả. Đồ án này đề xuất và xây dựng một hệ thống quản lý và chia sẻ tài liệu học tập có tổ chức, hỗ trợ người dùng ở nhiều cấp độ khác nhau (cơ bản, nâng cao, chuyên sâu). Hệ thống cho phép người dùng đăng tải, tìm kiếm, truy cập và tải về tài liệu một cách dễ dàng, đồng thời có cơ chế phân quyền để kiểm soát truy cập phù hợp với nhu cầu và trình độ người dùng. Giải pháp này nhằm tối ưu hóa quá trình học tập, chia sẻ tri thức và nâng cao hiệu quả khai thác tài liệu số.
	
	\vspace{2cm}
	
	\section*{ABSTRACT}
	In today's rapidly digitalized world, the demand for accessing and utilizing educational and research materials is steadily increasing. However, the fragmentation of resources across various platforms, along with a lack of structure and appropriate access control, often hinders users from efficiently finding and using valuable content. This project introduces and develops an organized document management and sharing system that supports users at different proficiency levels (basic, intermediate, advanced). The system enables users to upload, search, access, and download documents with ease, while implementing access control mechanisms to ensure that materials are delivered according to users’ needs and expertise levels. This solution aims to optimize the learning process, facilitate knowledge sharing, and enhance the efficiency of digital resource utilization.
	\cleardoublepage
	
	\pagenumbering{arabic} % Đánh số thứ tự 1,2,3...
	\section*{CHƯƠNG 1. TỔNG QUAN ĐỀ TÀI}
	\addcontentsline{toc}{section}{\numberline{}CHƯƠNG 1. TỔNG QUAN ĐỀ TÀI}
	\setcounter{section}{1}
	\setcounter{subsection}{0}
	\setcounter{figure}{0}
	\setcounter{table}{0}
	
	Chương này sẽ trình bày về tổng quan tình hình nghiên cứu, lý do chọn đề tài,phương pháp được chọn để thực hiện đề tài, mục tiêu đề tài cần phải đạt được.
	
	\subsection{Đặt vấn đề}
	
	Trong những năm gần đây, cùng với sự phát triển mạnh mẽ của công nghệ thông tin và Internet, quá trình học tập và nghiên cứu đang dần dịch chuyển sang môi trường số hóa. Tài liệu học tập, tài liệu nghiên cứu và các nguồn tri thức khác được số hóa và lưu trữ ngày càng nhiều trên các nền tảng trực tuyến. Điều này mang lại cơ hội lớn cho người học và người làm công tác nghiên cứu tiếp cận nhanh chóng và tiện lợi với kho tri thức toàn cầu.
	
	Tuy nhiên, bên cạnh những thuận lợi, thực tế cũng đặt ra nhiều thách thức. Khối lượng tài liệu số ngày càng đồ sộ, nhưng lại thiếu sự tổ chức hợp lý, gây khó khăn cho việc tra cứu, chọn lọc và sử dụng hiệu quả. Nhiều hệ thống tài liệu hiện tại chỉ phục vụ trong nội bộ (ví dụ như thư viện số của một trường đại học), trong khi các hệ thống mở lại thiếu tính kiểm soát và phân loại rõ ràng. 
	
	Chính vì vậy, việc xây dựng một hệ thống hỗ trợ quản lý, chia sẻ và truy cập tài liệu học tập dễ sử dụng là rất cần thiết. Hệ thống cần giải quyết đồng thời nhiều yêu cầu như: cho phép người dùng đóng góp tài liệu, tìm kiếm nhanh chóng. Điều này không chỉ giúp tiết kiệm thời gian và công sức trong việc tra cứu tài liệu, mà còn góp phần nâng cao chất lượng học tập, nghiên cứu và chia sẻ tri thức một cách có định hướng.
	
	Từ những lý do trên, nhóm thực hiện đồ án đã lựa chọn đề tài: \textbf{\textit{"Xây dựng website hỗ trợ học tập"}} nhằm giải quyết những vấn đề thực tiễn đang tồn tại, đồng thời đóng góp một công cụ hữu ích cho cộng đồng học tập và nghiên cứu.
	
	\subsection{Mục đích nghiên cứu}
	
	Mục đích chính của đề tài là xây dựng một hệ thống quản lý và chia sẻ tài liệu học tập có tổ chức, hiệu quả và dễ sử dụng, nhằm hỗ trợ người dùng ở nhiều cấp độ (từ cơ bản đến nâng cao) trong việc tiếp cận và khai thác tài nguyên tri thức một cách thuận tiện, nhanh chóng và phù hợp với nhu cầu.
	
	Cụ thể, đề tài hướng đến các mục tiêu sau:

	\begin{itemize}
		\item Thiết kế và phát triển hệ thống web hỗ trợ lưu trữ, phân loại, tìm kiếm và chia sẻ tài liệu học tập, nghiên cứu hoặc chuyên ngành một cách khoa học và dễ sử dụng.
		\item Tạo điều kiện cho người dùng đóng góp tài liệu, với quy trình kiểm duyệt hợp lý nhằm đảm bảo chất lượng nội dung, tính chính xác và phù hợp với định hướng học thuật.
		\item Góp phần xây dựng một nền tảng học tập mở nhưng có tổ chức, giúp nâng cao hiệu quả tự học, tiết kiệm thời gian tìm kiếm tài liệu, đồng thời tạo môi trường học thuật cộng tác và chia sẻ kiến thức lành mạnh.
	\end{itemize}
	
	\subsection{Kết luận chương}
	
	Tóm lại, trước những bất cập trong việc truy cập và sử dụng tài liệu hiện nay, việc xây dựng một hệ thống hỗ trợ, quản lý tài liệu học tập là hết sức cần thiết. Đề tài đã xác định được mục tiêu, phạm vi và định hướng phát triển hệ thống nhằm hỗ trợ người dùng hiệu quả hơn trong quá trình học tập và nghiên cứu. Để triển khai hệ thống một cách khoa học, chương tiếp theo sẽ trình bày các cơ sở lý thuyết liên quan làm nền tảng cho quá trình phân tích và thiết kế hệ thống.
	\newpage
	
	\section*{CHƯƠNG 2. CƠ SỞ LÝ THUYẾT}
	\addcontentsline{toc}{section}{\numberline{}CHƯƠNG 2. CƠ SỞ LÝ THUYẾT}
	\setcounter{section}{2}
	\setcounter{subsection}{0}
	\setcounter{figure}{0}
	\setcounter{table}{0}
	
	Chương này sẽ trình bày những kiến thức nền tảng liên quan đến hệ thống quản lý tài liệu, phân quyền người dùng, mô hình kiến trúc phần mềm, cũng như các công nghệ được sử dụng trong việc thiết kế và phát triển hệ thống. Những cơ sở lý thuyết này sẽ giúp đảm bảo hệ thống được triển khai đúng hướng, đáp ứng tốt các yêu cầu đặt ra về chức năng, hiệu năng và bảo mật.
	
	\subsection{Mô hình MVC}
	
	Mô hình MVC (Model-View-Controller) là một mẫu kiến trúc phần mềm phổ biến được sử dụng trong phát triển ứng dụng web, giúp phân chia ứng dụng thành ba thành phần chính: Model, View, và Controller được biểu diễn trên Hình \ref{fig21}. Mỗi thành phần đảm nhận một vai trò riêng, giúp tăng tính tổ chức, khả năng bảo trì và mở rộng của ứng dụng. 
	
	\begin{figure}[!ht]
		\centering
		\includegraphics[trim= 10pt 10pt 10pt 10pt, clip, width=8cm]{mvc_fig21.pdf}
		\caption[Mô hình MVC]{\bfseries \fontsize{12pt}{0pt}\selectfont Mô hình MVC}
		\label{fig21}
	\end{figure}
	
	\textbf{Model:} Đây là thành phần chịu trách nhiệm xử lý dữ liệu và logic nghiệp vụ. Model quản lý trạng thái của dữ liệu, tương tác với cơ sở dữ liệu (chẳng hạn như MySQL, SQL Server, SQLite, ...), và thực hiện các thao tác như thêm, sửa, xóa hoặc truy vấn thông tin tài liệu, người dùng, v.v...
	
	\textbf{View:} Thành phần này đại diện cho giao diện người dùng - nơi người dùng tương tác trực tiếp với hệ thống. Trong đề tài, phần View được phát triển bằng ReactJS, hiển thị các dữ liệu như danh sách tài liệu, thông tin chi tiết, giao diện đăng nhập, đăng ký, phân quyền, v.v... View nhận dữ liệu từ Controller và hiển thị lại cho người dùng theo cách trực quan, dễ sử dụng.
	
	\textbf{Controller:} Controller đóng vai trò trung gian giữa Model và View. Nó tiếp nhận các yêu cầu từ người dùng thông qua View (chẳng hạn như thao tác đăng nhập, tìm kiếm tài liệu, tải lên file), xử lý yêu cầu đó bằng cách tương tác với Model, sau đó trả kết quả lại cho View để hiển thị. Trong hệ thống này, Controller được cài đặt ở backend thông qua Node.js và Express, đóng vai trò định tuyến và xử lý logic của ứng dụng.
	
	Việc áp dụng mô hình MVC giúp hệ thống được chia tách rõ ràng theo chức năng, giúp dễ dàng mở rộng, tái sử dụng mã nguồn, và thuận tiện cho việc phát triển theo nhóm hoặc bảo trì lâu dài. Đây là một kiến trúc phù hợp cho các ứng dụng web hiện đại, đặc biệt là các hệ thống có quy mô vừa và lớn.
	
	\subsection{Ngôn ngữ lập trình JavaScript}
	
	JavaScript là một ngôn ngữ lập trình thông dịch, hướng đối tượng, và là một trong ba công nghệ cốt lõi của phát triển web cùng với HTML và CSS. Ban đầu JavaScript được thiết kế để xử lý các tương tác trên trình duyệt phía client (frontend), nhưng với sự phát triển của các công nghệ hiện đại như Node.JS, JavaScript ngày nay còn được sử dụng mạnh mẽ ở phía server (backend), trở thành một ngôn ngữ lập trình toàn diện cho cả hai phía.
	
	Trong đề tài, JavaScript đóng vai trò ngôn ngữ lập trình chính cho cả frontend và backend:
	
	Ở phía frontend, JavaScript được sử dụng kết hợp với thư viện ReactJS để xây dựng giao diện người dùng (UI) tương tác, phản hồi nhanh, và cập nhật linh hoạt theo trạng thái dữ liệu.
	
	Ở phía backend, JavaScript được sử dụng trong môi trường Node.js để xây dựng API, xử lý yêu cầu từ client, truy vấn cơ sở dữ liệu, và quản lý logic nghiệp vụ.
	
	Các ưu điểm chính của JavaScript trong dự án bao gồm:
	\begin{itemize}
		\item Tính nhất quán về ngôn ngữ giữa frontend và backend, giúp đồng bộ logic xử lý và giảm chi phí.
		\item Cộng đồng phát triển mạnh, hỗ trợ thư viện phong phú (npm), dễ mở rộng và tích hợp.
		\item Tốc độ thực thi nhanh, đặc biệt với V8 Engine – giúp backend hoạt động hiệu quả.
		\item Khả năng xử lý bất đồng bộ tốt (với async/await, promise), rất phù hợp với ứng dụng web có nhiều tác vụ I/O như đọc/ghi file, truy vấn database, v.v...
	\end{itemize}
	
	Nhờ sử dụng JavaScript xuyên suốt hệ thống, đề tài có thể phát triển một cách linh hoạt, đồng bộ và dễ bảo trì, đồng thời tận dụng được các công cụ hiện đại hỗ trợ phát triển web hiệu quả hơn.
	
	\subsection{Node.js}
	
	Node.js là một môi trường chạy JavaScript phía máy chủ (server-side), được xây dựng trên nền tảng của Google V8 Engine – trình biên dịch JavaScript hiệu năng cao được sử dụng trong trình duyệt Chrome. Node.js cho phép lập trình viên sử dụng JavaScript để viết các ứng dụng phía backend, từ đó thống nhất ngôn ngữ lập trình giữa frontend và backend.
	
	Một trong những điểm mạnh lớn nhất của Node.js là kiến trúc non-blocking I/O và event-driven (dựa trên sự kiện), giúp xử lý hàng loạt yêu cầu đồng thời một cách hiệu quả mà không làm nghẽn luồng xử lý chính. Điều này đặc biệt phù hợp với các ứng dụng web thời gian thực, ứng dụng cần xử lý nhiều kết nối đồng thời hoặc hệ thống nhẹ mà hiệu năng cao.
	
	Trong khuôn khổ đề tài, Node.js được sử dụng để:
	\begin{itemize}
		\item Xây dựng máy chủ backend phục vụ cho API xử lý logic nghiệp vụ và phản hồi dữ liệu cho frontend.
		\item Tích hợp với framework Express.js nhằm đơn giản hóa việc định tuyến, xử lý yêu cầu và triển khai RESTful API.
		\item Giao tiếp với cơ sở dữ liệu MySQL để lưu trữ và truy xuất thông tin người dùng, tài liệu, và phân quyền.
		\item Xử lý đăng nhập, xác thực, phân quyền truy cập, và quản lý file upload một cách linh hoạt.
	\end{itemize}
	
	Việc sử dụng Node.js giúp hệ thống đạt được hiệu suất cao, giảm độ trễ phản hồi, đồng thời dễ dàng mở rộng và bảo trì nhờ vào cộng đồng phát triển lớn và hệ sinh thái thư viện phong phú (thông qua npm – Node Package Manager).
	
	\subsection{MySQL}
	
	MySQL là một hệ quản trị cơ sở dữ liệu quan hệ (Relational Database Management System – RDBMS) mã nguồn mở, được phát triển và duy trì bởi Oracle Corporation. MySQL sử dụng ngôn ngữ truy vấn SQL (Structured Query Language) để thao tác với dữ liệu, và được sử dụng rộng rãi trong các ứng dụng web nhờ vào khả năng ổn định, hiệu năng tốt và dễ triển khai.
	
	Trong hệ thống quản lý tài liệu học tập của đề tài, MySQL đóng vai trò là cơ sở dữ liệu trung tâm, chịu trách nhiệm lưu trữ và tổ chức toàn bộ dữ liệu của hệ thống.
	
	MySQL hỗ trợ quan hệ giữa các bảng, nhờ đó có thể xây dựng mô hình dữ liệu chặt chẽ và logic, đảm bảo tính toàn vẹn và dễ truy vấn. Trong đề tài, hệ thống sử dụng các thao tác phổ biến như:
	\begin{itemize}
		\item SELECT để truy xuất dữ liệu (tìm kiếm tài liệu, thông tin người dùng).
		\item INSERT, UPDATE, DELETE để thao tác thêm/sửa/xóa dữ liệu.
		\item JOIN để kết nối các bảng liên quan, ví dụ giữa bảng người dùng và bảng tài liệu.
		\item Thiết lập khóa chính – khóa ngoại (Primary Key – Foreign Key) để đảm bảo ràng buộc dữ liệu.
	\end{itemize}
	
	MySQL còn được đánh giá cao vì khả năng tương thích tốt với Node.js thông qua các thư viện như mysql2 hoặc sequelize, cho phép backend giao tiếp với cơ sở dữ liệu một cách nhanh chóng và hiệu quả.
	
	Việc lựa chọn MySQL giúp hệ thống đảm bảo khả năng mở rộng, dễ quản trị và phù hợp với quy mô của ứng dụng web hiện đại hướng đến lưu trữ nhiều tài liệu và người dùng.
	
	\subsection{ReactJS}
	
	ReactJS là một thư viện JavaScript mã nguồn mở được phát triển bởi Facebook, dùng để xây dựng giao diện người dùng (User Interface – UI) cho các ứng dụng web. React hoạt động theo mô hình component-based (thành phần hóa), cho phép chia nhỏ giao diện thành các phần độc lập có thể tái sử dụng, giúp tăng tính linh hoạt, dễ bảo trì và mở rộng hệ thống.
	
	Một trong những điểm nổi bật của React là cơ chế Virtual DOM – giúp tối ưu hiệu suất hiển thị bằng cách chỉ cập nhật các phần tử thay đổi trên giao diện, thay vì render lại toàn bộ trang. Điều này giúp tăng tốc độ phản hồi và cải thiện trải nghiệm người dùng.
	
	Trong đề tài, ReactJS được sử dụng để xây dựng toàn bộ phần giao diện người dùng (frontend), bao gồm các trang như đăng nhập, đăng ký, danh sách tài liệu, form đóng góp tài liệu và phân quyền truy cập. Giao diện này kết nối với backend thông qua các RESTful API được xây dựng bằng Node.js, cho phép gửi và nhận dữ liệu một cách linh hoạt. Bên cạnh đó, React còn hỗ trợ quản lý trạng thái dữ liệu và điều hướng giữa các trang bằng các thư viện tích hợp như React Router, useState, useEffect, giúp giao diện hoạt động mượt mà và phản hồi nhanh chóng.
	
	Các ưu điểm chính khi sử dụng React trong dự án:
	\begin{itemize}
		\item Giao diện hiện đại, tương tác mượt mà và linh hoạt.
		\item Dễ mở rộng và bảo trì nhờ kiến trúc component rõ ràng.
		\item Cộng đồng lớn, hỗ trợ nhiều thư viện và công cụ đi kèm.
		\item Kết hợp tốt với các công nghệ backend như Node.js, Express.
	\end{itemize}
	
	Việc sử dụng React giúp hệ thống không chỉ thân thiện với người dùng mà còn dễ phát triển thêm các tính năng mới, phù hợp với xu hướng phát triển ứng dụng web hiện đại.
	\newpage
	\section*{CHƯƠNG 3. PHÂN TÍCH VÀ THIẾT KẾ HỆ THỐNG}
	\addcontentsline{toc}{section}{\numberline{}CHƯƠNG 3. PHÂN TÍCH VÀ THIẾT KẾ HỆ THỐNG}
	\setcounter{section}{3}
	\setcounter{subsection}{0}
	\setcounter{figure}{0}
	\setcounter{table}{0}
	
	Để xây dựng một hệ thống phần mềm hiệu quả, có tính ứng dụng thực tiễn cao và đáp ứng đúng nhu cầu người dùng, quá trình phân tích và thiết kế hệ thống là bước không thể thiếu. Chương này trình bày các nội dung phân tích và thiết kế cốt lõi nhằm làm rõ phạm vi bài toán, luồng xử lý dữ liệu, chức năng hệ thống cũng như mô hình hóa các thành phần trong hệ thống dưới dạng sơ đồ trực quan.
	
	\subsection{Sơ đồ phân cấp chức năng (FHD)}
	
	Dựa vào các yêu cầu chức năng, ta lập được sơ đồ phân cấp chức năng của hệ thống như Hình \ref{fig31}.
	
	\begin{figure}[!ht]
		\centering
		\includegraphics[trim= 10pt 10pt 10pt 10pt, clip, width=16cm]{fhd_fig31_2.pdf}
		\caption [Sơ đồ phân cấp chức năng hệ thống]{\bfseries \fontsize{12pt}{0pt}\selectfont Sơ đồ phân cấp chức năng hệ thống}
		\label{fig31}
	\end{figure}
	
	Hệ thống hỗ trợ học tập được phân chia thành ba phân hệ chính, mỗi phân hệ bao gồm các chức năng cụ thể như sau:
	
	\textbf{Quản trị hệ thống:} Cung cấp các chức năng liên quan đến quản lý người dùng và tài khoản. Người dùng có thể thực hiện đăng ký, chỉnh sửa thông tin cá nhân, thay đổi mật khẩu và được phân quyền truy cập tương ứng theo vai trò (cơ bản, plus, premium). Ngoài ra, hệ thống hỗ trợ giám sát hoạt động của tài khoản và thực hiện các thao tác quản lý hệ thống một cách toàn diện.
	
	\textbf{Quản lý cơ sở dữ liệu về môn học:} Cho phép người dùng quản lý danh sách các môn học, bao gồm thêm mới, chỉnh sửa hoặc xóa thông tin học phần. Hệ thống hỗ trợ tính năng tìm kiếm theo từ khóa và mã môn học nhằm thuận tiện cho việc tra cứu. Ngoài ra, còn có chức năng thống kê số lượng người dùng đã đăng ký học phần và thống kê người dùng đã tạo môn học.
	
	\textbf{Quản lý cơ sở dữ liệu về tài liệu học tập:} Hỗ trợ người dùng quản lý và khai thác hệ thống tài liệu liên quan đến các môn học. Người dùng có thể đăng tải tài liệu mới, chỉnh sửa thông tin tài liệu đã có hoặc tìm kiếm tài liệu theo tên, loại tài liệu hoặc khóa học liên quan. Hệ thống cũng thống kê được số lượng người dùng đã đóng góp tài liệu nhằm đánh giá mức độ tương tác và chia sẻ trong cộng đồng.
	
	\subsection{Sơ đồ luồng dữ liệu (DFD)}
	
	\subsubsection{Mức ngữ cảnh}
	
	Sơ đồ mức ngữ cảnh của hệ thống được mô tả như trên Hình \ref{fig32}. Tiến trình của hệ thống nằm trong mối quan hệ hai thực thể ngoài là: quản trị viên và người dùng. Cụ thể các luồng dữ liệu như sau:
	\begin{figure}[!ht]
		\centering
		\includegraphics[trim= 10pt 10pt 10pt 10pt, clip, width=15cm]{dfd_fig32.pdf}
		\caption [Sơ đồ luồng dữ liệu mức ngữ cảnh]{\bfseries \fontsize{12pt}{0pt}\selectfont Sơ đồ luồng dữ liệu mức ngữ cảnh}
		\label{fig32}
	\end{figure}
	
	(1) Thông tin các cơ sở dữ liệu liên quan đến người dùng, môn học và tài liệu.

	(2) Tra cứu, quản lý các cơ sở dữ liệu người dùng, môn học và tài liệu.
	
	(3) Thông tin người dùng.
	
	(4) Tra cứu, quản lý người dùng qua các thao tác như thay đổi họ tên, ngày sinh và thay đổi mật khẩu.
	
	\subsubsection{Mức đỉnh}
	
	Sơ đồ luồng dữ liệu mức đỉnh được thể hiện trên Hình \ref{fig33}. Mô hình này có:
	
	\begin{itemize}
		\item Chức năng: Quản trị hệ thống; Quản lý cơ sở dữ liệu về môn học; Quản lý cơ sở dữ liệu về tài liệu học tập.
		\item Tác nhân bên ngoài: Quản trị viên, người dùng.
		\item Kho dữ liệu: a, b.
	\end{itemize}
	
	\begin{figure}[!ht]
		\centering
		\includegraphics[trim= 10pt 10pt 10pt 10pt, clip, width=15cm]{mucdinh_fig33.pdf}
		\caption [Sơ đồ luồng dữ liệu mức đỉnh]{\bfseries \fontsize{12pt}{0pt}\selectfont Sơ đồ luồng dữ liệu mức đỉnh}
		\label{fig33}
	\end{figure}
	
	\subsubsection{Mức dưới đỉnh}
	\paragraph{Chức năng quản trị hệ thống} \mbox{}
	
	\begin{figure}[!ht]
		\centering
		\includegraphics[trim= 10pt 10pt 10pt 10pt, clip, width=15cm]{mucduoidinh_fig34.pdf}
		\caption [Sơ đồ luồng dữ liệu mức dưới đỉnh của chức năng quản trị hệ thống]{\bfseries \fontsize{12pt}{0pt}\selectfont Sơ đồ luồng dữ liệu mức dưới đỉnh của chức năng quản trị hệ thống}
		\label{fig34}
	\end{figure}
	
	Sơ đồ mức dưới đỉnh của chức năng quản trị hệ thống được mô tả trên Hình \ref{fig34}.
	
	\paragraph{Chức năng quản lý CSDL về môn học} \mbox{}
	
	\begin{figure}[!ht]
		\centering
		\includegraphics[trim= 10pt 10pt 10pt 10pt, clip, width=16.25cm]{mucduoidinh_fig35.pdf}
		\caption [Sơ đồ luồng dữ liệu mức dưới đỉnh của chức năng quản lý CSDL về môn học]{\bfseries \fontsize{12pt}{0pt}\selectfont Sơ đồ luồng dữ liệu mức dưới đỉnh của chức năng quản lý CSDL về môn học}
		\label{fig35}
	\end{figure}
	
	Sơ đồ mức dưới đỉnh của chức năng quản lý CSDL về môn học được mô tả trên Hình \ref{fig35}.
	
	\paragraph{Chức năng quản lý CSDL về tài liệu môn học} \mbox{}
	
	\begin{figure}[!ht]
		\centering
		\includegraphics[trim= 10pt 10pt 10pt 10pt, clip, width=16.25cm]{mucduoidinh_fig36.pdf}
		\caption [Sơ đồ luồng dữ liệu mức dưới đỉnh của chức năng quản lý CSDL về tài liệu]{\bfseries \fontsize{12pt}{0pt}\selectfont Sơ đồ luồng dữ liệu mức dưới đỉnh của chức năng quản lý CSDL về tài liệu}
		\label{fig36}
	\end{figure}
	
	Sơ đồ mức dưới đỉnh của chức năng quản lý CSDL về tài liệu học tập được mô tả trên Hình \ref{fig36}.
	
	\subsection{Sơ đồ kịch bản sử dụng (Use Case)}
	
	Hệ thống sẽ bao gồm 3 tác nhân sử dụng hệ thống:
	
	\textbf{Admin:} Là người quản trị toàn hệ thống, có quyền truy cập và thao tác với tất cả các chức năng. Admin có thể quản lý người dùng (xem, tìm kiếm, thêm, chỉnh sửa, xóa tài khoản), phân quyền truy cập, quản lý các môn học, tài liệu và thống kê toàn hệ thống.
	
	\textbf{User:} Có quyền truy cập vào các chức năng cơ bản của hệ thống như xem và tìm kiếm tài liệu, tra cứu thông tin môn học, xem mô tả khóa học, và quản lý thông tin cá nhân của chính mình.
	
	\subsubsection{Danh sách Use Case}
	
		(1) Đăng nhập/Đăng xuất.
		
		(2) Đăng kí (dành cho người dùng mới).
		
		(3) Đăng kí môn học.
		
		(4) Xem thông tin môn học.
		
		(5) Lưu môn học.
		
		(6) Hiển thị và tải tài liệu.
		
		(7) Hiển thị các môn học đã đăng kí.
		
		(8) Hiển thị các môn học đã lưu.
		
		(9) Hiển thị danh sách người dùng (dành cho quản trị viên).
		
		(10) Xóa người dùng (dành cho quản trị viên).
	
	\subsubsection{Sơ đồ Use Case}
\end{document}